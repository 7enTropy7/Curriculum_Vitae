\documentclass[a4paper, 12pt]{article}
\usepackage[top=0.5in,bottom=0.5in,left=0.5in,right=0.5in]{geometry}
\usepackage[utf8]{inputenc}
\usepackage[hidelinks]{hyperref}
\usepackage{xcolor}
\usepackage{setspace}
\usepackage{fontawesome5}
\usepackage{multicol}

\renewcommand{\baselinestretch}{0.5}
\begin{document}
    \begin{center}
        \Huge{\textbf{UNNIKRISHNAN R. MENON}}
    \end{center}
    
    \noindent\rule{\textwidth}{1pt}
    \faEnvelope\href{menon.uk1998@gmail.com}{\textcolor{blue}{\,\underline{menon.uk1998@gmail.com}}}\qquad\qquad\qquad\qquad\qquad\faPhone\,+91\,$-$\,8376048185\\
        \faMapMarker* VIT University, Vellore, Tamil Nadu\,$-$\,632014\quad\,\,\faGithub\href{github.com/7enTropy7}{\textcolor{blue}{\,\,\underline{github.com/7enTropy7}}}\\
        \faLinkedin\,\,\href{linkedin.com/in/unnikrishnan-menon-aa013415a}{\textcolor{blue}{\underline{linkedin.com/in/unnikrishnan-menon-aa013415a}}}\,\,\,\faQuora\href{quora.com/profile/Unnikrishnan-Menon-5}{\textcolor{blue}{\,\underline{quora.com/profile/Unnikrishnan-Menon-5}}}\\
        \noindent\rule{\textwidth}{1pt}
        
        \section*{Education}
        \begin{flushleft}
        \textbf{B.Tech in Electrical and Electronics Engineering}\qquad\qquad\qquad\qquad 02/2017\,$-$\,Present\\
        Vellore Institute of Technology, Vellore\\ Current CGPA (3 semesters): 8.62
    \end{flushleft}
    
    \begin{flushleft}
        \textbf{Class 12 Board Examination (CBSE): 94.2\%}\qquad\qquad\qquad\qquad\qquad 2017\\Summer Fields School, New Delhi
    \end{flushleft}
    
    \begin{flushleft}
        \textbf{Class 10 Board Examination (CBSE): 10 CGPA}\qquad\qquad\qquad\qquad 2015\\Summer Fields School, New Delhi
    \end{flushleft}
    
    \section*{Research Interests}
    \begin{multicols}{2}
    	\begin{itemize}
    		\item Artifical Intelligence
    		\item Machine Learning
    		\item Cryptography
    		\item Computer Vision
    		\item Socket Programming
    		\item Astrophysics
    		\item Reinforcement Learning
    		\item Genetic Algorithm
    	\end{itemize}
    \end{multicols}

    \section*{Technical Skills}
	\begin{flushleft}
	    \begin{itemize}
	    	\item \textbf{Electronic Prototyping Platform}$\,-\,$ Arduino, Raspberry Pi, Microcontroller Model 8051
	        \item \textbf{Programming Languages}$\,-\,$Python, C++, C, Java, GoLang, Assembly x86 - MASM
	        \item \textbf{Mathematical Packages}$\,-\,$MATLAB, R
	        \item \textbf{Typesetting Software}$\,-\,$\LaTeX
	        \item \textbf{Other}$\,-\,$TensorFlow, Keras, OpenAI Gym, NumPy
	    \end{itemize}
	\end{flushleft}
	
	\section*{Work Experience}
    \begin{flushleft}
        02/2019\,$-$\,Present\\
        \textbf{Technical Head of Electrical Department}\\
        roboVITics, the official robotics club of VIT
    \end{flushleft}
    
    \begin{flushleft}
        07/2018\,$-$\,Present\\
        \textbf{High Power Circuit Designer}\\
        \textit{The team designed a 120 lbs combat robot that has performed well in international RoboWars}\\\textbf{Achievements}\\\begin{itemize}
            \item Finished in top 7 internationally at RoboWars, TechFest'1, IIT Bombay\\
            \item Secured third position in RoboWars, Kurukshetra'19, Anna University
        \end{itemize}
    \end{flushleft}
    
    \begin{flushleft}
        12/2017\,$-$\,02/2019\\
        \textbf{Core-Committee Member}\\
        roboVITics, the official robotics club of VIT
        \begin{itemize}
            \item Successfully completed multiple robotics projects involving Machine Learning, Computer Vision, Artificial Intelligence, IoT etc.
        \end{itemize}
    \end{flushleft}
	
	\section*{Achievements}
	\begin{flushleft}
        \textbf{6 Near Earth Object Observations (All India Asteroid Search Campaign)}\qquad 2016\\\textit{I used a software called Astrometrica to detect potential celestial objects and I ended up spotting 6 Near Earth Objects (NEO's)}
    \end{flushleft}
    
    \begin{flushleft}
        
        \textbf{Winner of Developer's Sprint of Code Hackathon by CodeChef}\qquad\qquad\qquad\quad 02/2019\\\textit{Secured the First Position in this 36 hour Hackathon. I worked on the hardware and a facial emotion recognizer for an electoral system that eliminates majority of the problems in the existing system}
    \end{flushleft}
    
    \begin{flushleft}
        \textbf{Quora Top Writer 2018}\qquad\qquad\qquad\qquad\qquad\qquad\qquad\qquad\qquad\qquad\qquad\qquad 01/2018\,$-$\,Present\\\textit{Got the coveted Top Writer's Quill on my Quora profile for writing quality technical content. Got New York Time's subscription and a t-shirt as a reward from Quora}
    \end{flushleft}
    
    \section*{Personal Projects}
    \begin{flushleft}
    	09$/$2019\,$-$\,10$/$2019\\
    	\textbf{Path Prediction for Smart Vehicles}
    	\begin{itemize}
    		\item A Path Prediction Algorithm which forecasts future path taken using RNN$-$LSTMs and on top
of that optimizes the predicted trajectory using Deep Q-Learning Algorithm.
    	\end{itemize}
    \end{flushleft}
    \begin{flushleft}
    	08$/$2019\,$-$Present\\
    	\textbf{Riff$-$Raff Encryption}
    	\begin{itemize}
    		\item Decimal (Negative/Positive,Unranged) Encryption for Unbreakable, Impenetrable Security.
    	\end{itemize}
    \end{flushleft}
    
    \begin{flushleft}
    	12$/$2018\,$-$\,02$/$2019\\
    	\textbf{Self Learning Crawler}
    	\begin{itemize}
    		\item This bot combines the Q Learning algorithm with a robotic arm to come up with an optimum
policy for moving forward
    	\end{itemize}
    \end{flushleft}
    
    \begin{flushleft}
    	05$/$2019\,$-$\,12$/$2019\\
    	\textbf{RSA Encrypted Password Online Storage}
    	\begin{itemize}
    		\item This code can be used to save your passwords or other confidential data remotely to a server with a layer of RSA encryption (coded from scratch) without any worries of it getting hacked.
    	\end{itemize}
    \end{flushleft}
    
    \begin{flushleft}
    	02$/$2019\,$-$\,02$/$2019\\
    	\textbf{Comprehensive Electoral Solution Suite}
    	\begin{itemize}
    		\item Secured First Position in DEVSOC'19
    	\end{itemize}
    \end{flushleft}
    
    \begin{flushleft}
    	01$/$2019\,$-$\,02$/$2019\\
    	\textbf{Prepaid Energy Credits based Power Distribution System}
    	\begin{itemize}
    		\item Machine learning based algorithm for predicting power usage in a common household.
    	\end{itemize}
    \end{flushleft}    
    
    \begin{flushleft}
    	10$/$2018\,$-$Present\\
    	\textbf{AI Development for Video Games}
    	\begin{itemize}
    		\item Deployed genetic algorithms and other advanced reinforcement learning algorithms in various
Video Game environments like Super Mario, Pacman, Snakes, Flappy Birds etc.
    	\end{itemize}
    \end{flushleft}
    
    \begin{flushleft}
    	05$/$2018\,$-$\,06$/$2018\\
    	\textbf{TensorFlow ChatBot}
    	\begin{itemize}
    		\item An RNN and LSTM based Chatbot that responds well to meaningful queries.
    	\end{itemize}
    \end{flushleft}
    
    \begin{flushleft}
    	02$/$2018\,$-$\,04$/$2018\\
    	\textbf{Autonomous Rubik's Cube Solver}
    	\begin{itemize}
    		\item Developed an algorithm in under $800$ lines of C++ code that predicts the correct moves to solve
a scrambled $3\times3\times3$ Rubik's Cube.
    	\end{itemize}
    \end{flushleft}
    
    \pagebreak
    \section*{References}
    \begin{flushleft}
    \textbf{Awnon Bhowmik, College Laboratory Technician}\\
    Department of Mathematics\\
    CUNY Borough of Manhattan Community College\\
    +1 (929) 462 8832, \href{abhowmik@bmcc.cuny.edu}{\textcolor{blue}{\underline{abhowmik@bmcc.cuny.edu}}}
    \end{flushleft}
\end{document}