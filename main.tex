\documentclass[a4paper, 12pt]{article}
\usepackage[top=0.5in,bottom=0.5in,left=0.5in,right=0.5in]{geometry}
\usepackage[utf8]{inputenc}
\usepackage{hyperref}
\usepackage{xcolor}
\usepackage{setspace}

\renewcommand{\baselinestretch}{0.5}
\begin{document}
    \begin{center}
        \Huge{\textbf{UNNIKRISHNAN MENON}}
    \end{center}
    
    \begin{center}
        \textbf{Electrical and Electronics Undergrad}
    \end{center}
    
    \begin{flushleft}
    \noindent\rule{\textwidth}{1pt}\\
        Email:\href{menon.uk1998@gmail.com}{\textcolor{blue}{\,\,\,\,\,\,\, \underline{menon.uk1998@gmail.com}}}\qquad\qquad\qquad\qquad\qquad\qquad\qquad Phone: +91-8376048185\\
        \small{Address:\,\,\, VIT University, Vellore, Tamil Nadu - 632014, Vellore, India}\quad GitHub:\href{github.com/7enTropy7}{\textcolor{blue}{\,\,\underline{github.com/7enTropy7}}}\\LinkedIn: \href{linkedin.com/in/unnikrishnan-menon-aa013415a}{\textcolor{blue}{\,\,\underline{linkedin.com/in/unnikrishnan-menon-aa013415a}}}\\Quora:\href{quora.com/profile/Unnikrishnan-Menon-5}{\textcolor{blue}{\qquad\underline{quora.com/profile/Unnikrishnan-Menon-5}}}\\
        \noindent\rule{\textwidth}{1pt}
    \end{flushleft}
    
    \begin{flushleft}
        \textbf{WORK EXPERIENCE}\\\noindent\rule{\textwidth}{1pt}\\
        02/2019\,$-$\,Present\\
        \textbf{Technical Head of Electrical Department}\\
        roboVITics, the official robotics club of VIT
    \end{flushleft}
    
    \begin{flushleft}
        07/2018\,$-$\,Present\\
        \textbf{High Power Circuit Designer}\\
        \textit{The team designed a 120 lbs combat robot that has performed well in international RoboWars}\\\textbf{Achievements}\\\begin{itemize}
            \item Finished in top 7 internationally at RoboWars, TechFest'1, IIT Bombay\\
            \item Secured third position in RoboWars, Kurukshetra'19, Anna University
        \end{itemize}
    \end{flushleft}
    
    \begin{flushleft}
        12/2017\,$-$\,02/2019\\
        \textbf{Core-Committee Member}\\
        roboVITics, the official robotics club of VIT
        \begin{itemize}
            \item Successfully completed multiple robotics projects involving Machine Learning, Computer Vision, Artificial Intelligence, IoT etc.
        \end{itemize}
    \end{flushleft}

    \begin{flushleft}
        \textbf{EDUCATION}\\\noindent\rule{\textwidth}{1pt}\\
        02/2017\,$-$\,Present\\
        \textbf{B.Tech in Electrical and Electronics Engineering}\\
        Vellore Institute of Technology, Vellore\\ Current CGPA (3 semesters): 8.62
    \end{flushleft}
    
    \begin{flushleft}
        \textbf{Class 12 Board Examination (CBSE): 94.2\%}\\Summer Fields School, New Delhi
    \end{flushleft}
    
    \begin{flushleft}
        \textbf{Class 10 Board Examination (CBSE): 10 CGPA}\\Summer Fields School, New Delhi
    \end{flushleft}
    
    \begin{flushleft}
        \textbf{SKILLS}\\\noindent\rule{\textwidth}{1pt}\\Machine Learning, Artificial Intelligence, Cryptography, Astrophysics, Python, Competitive Coding, Combat Robotics, Electronics, Internet of Things, Robotics, C/C++, Java, MATLAB, Computer Vision, Back-end Development, Socket Programming
    \end{flushleft}
    
    \begin{flushleft}
        \textbf{ACHIEVEMENTS}\\\noindent\rule{\textwidth}{1pt}\\\textbf{6 Near Earth Object Observations (All India Asteroid Search Campaign)}\\\textit{I used a software called Astrometrica to detect potential celestial objects and I ended up spotting 6 Near Earth Objects (NEO's)}
    \end{flushleft}
    
    \begin{flushleft}
        02/2019\\
        \textbf{Winner of Developer's Sprint of Code Hackathon by CodeChef}\\\textit{Secured the First Position in this 36 hour Hackathon. I worked on the hardware and a facial emotion recognizer for an electoral system that eliminates majority of the problems in the existing system}
    \end{flushleft}
    
    \begin{flushleft}
        01/2018\,$-$\,Present\\
        \textbf{Quora Top Writer 2018}\\\textit{Got the coveted Top Writer's Quill on my Quora profile for writing quality technical content. Got New York Time's subscription and a t-shirt as a reward from Quora}
    \end{flushleft}
    
    \pagebreak
    
    \begin{flushleft}
        \textbf{PERSONAL PROJECTS}\\\noindent\rule{\textwidth}{1pt}\\02/2018\,$-$\,04/2018\\\textbf{Autonomous Rubik's Cube Solver}\\\begin{itemize}
            \item Developed an algorithm in under 800 lines of C++ code that predicts the correct moves to solve a scrambled $3\times3\times3$ Rubik's Cube.
        \end{itemize}
        
        12/2018\,$-$\,02/2019\\
        \textbf{Self Learning Crawler}\\\begin{itemize}
            \item This bot combines the Q Learning algorithm with a robotic arm to come up with an optimum policy for moving forward
        \end{itemize}
        
        02/2019\,$-$\,02/2019\\
        \textbf{Comprehensive Electoral System}\\\begin{itemize}
            \item Secured First Position in DEVSOC'19
        \end{itemize}
        
        01/2019\,$-$\,02/2019\\
        \textbf{Prepaid Energy Credits based Power Distribution System}\\\begin{itemize}
            \item ML based algorithm for predicting power usage in a common household.
        \end{itemize}
        
        05/2019\,$-$\,12/2019\\
        \textbf{RSA Encrypted Online Password Storage}\begin{itemize}
            \item This code can be used to save your passwords or other confidential data remotely to a server without any worries of it getting hacked.
        \end{itemize}
        
        05/2018\,$-$\,06/2018\\
        \textbf{TensorFlow Chatbot using RNN's and LSTM's}
    \end{flushleft}
    
    Add some of your other shit to fill this page since I've occupied it.
    \begin{flushleft}
        \textbf{REFERENCES}\\\noindent\rule{\textwidth}{1pt}\\\textbf{Awnon Bhowmik, College Laboratory Technician}\\Department of Mathematics\\Borough of Manhattan Community College\\(929) 462\,$-$\,8832,\href{mailto: abhowmik@bmcc.cuny.edu}{\textcolor{blue}{\,\underline{abhowmik@bmcc.cuny.edu}}}
    \end{flushleft}
\end{document}
